\documentclass[12pt,a4paper]{report}
\usepackage[utf8]{inputenc}
\usepackage[T1]{fontenc}
\usepackage{amsmath}
\usepackage{amssymb}
\usepackage{makeidx}
\usepackage{graphicx}
\usepackage{float}
\usepackage{gensymb}
\usepackage{lmodern,textcomp}
\title{Temperature Measurement of An FDM 3D Printer Hotend with a PTC
Thermistor}
\author{Metin Oktay Yılmaz}
\begin{document}
    \maketitle

    \section*{Introduction}
    An FDM 3D printer can utilize a variety of plastics ranging from PLA to
    ABS, even TPU. Succesful printing requires a stable temperature at the
    hotend and build plate. This study will investigate the feasibility of a
    possible feedback control setup for hotend temperature control.

    \section*{Setup}
    Following is the moderately-detailed flow diagram of control of a 3D
    printer hotend:
    \begin{figure}[H]
        \centering
        \includegraphics[scale=0.4]{diagram.png}
        \caption{The feedback loop that controls a hotend's temperature}
    \end{figure}

    The feedback loop works as follows:
    \begin{itemize}
        \item[1.] The user or the G-Code sets a target temperature for
        controller.
        \item[2.] Controller reads the hotend's temperature and calculates the
        required PWM duty cycle, then applies it to the FET.
        \item[3.] FET delivers electrical energy to the hotend. This could
        range from 0 Joules to as many as power supply can withhold. Hotend
        gains or loses temperature accordingly.
        \item[4.] Controller reads hotend's temperature again and the loop
        continues.
    \end{itemize}

    \section*{Requirement}

    A successful print requires a stable temperature at hotend. A temperature
    oscillation within the range of $\pm2\degree C$ is acceptable for a stable
    quality through printing and this is the requirement for this study:
    Implement a setup where the data acquisition capability allows a
    temperature control within $\pm2\degree C$ range of setpoint. Also, the
    budget of the project is 10\texteuro.

    \section*{Approach}

    Firstly, the material properties of hotend and its thermal behavior will be
    investigated and required control parameters will be determined. Then,
    controller and FET will be chosen accordingly.

    \section*{Outcome Estimation}

    Given the latest developments in 32-bit controllers and FETs, it is
    expected that this feasiblity study will return a green light. For example,
    the latest STM32C0 family MCUs provide 32-bit functionality at 48MHz for
    as low as 1.6 \$/unit, even they may be an overkill for the project.

\end{document}