\section*{ATMega328PB Characterisitics}
\addcontentsline{toc}{section}{ATMega328PB Characterisitics}

Originally made by Italian company Atmel, which later on has been purchased by Microchip Technologies in 2016 \cite{atmel}, Atmega328 line of MCUs came into the scene with the rise of Arduino \cite{arduino}. It is ubiquitous and fairly affordable with a price tag of 3.41 \$ per unit, and fairly well input-output capabilities make it a good candidate for this study.

Following data of MCU's ADC is a direct excerpt from the its datasheet \cite{ATMEGA328PB}:

\begin{enumerate}
    \item  10-bit Resolution
    \item  Rounding based schema\footnote{This was determined indirectly. There was no matching word in the datasheet for \textit{"truncat"}; therefore it has been assumed that the ADC is rounding-based.}
    \item  ±2 LSB Absolute Accuracy
    \item  Up to 76.9 kSPS (Up to 15 kSPS at Maximum Resolution)
    \item  Temperature Sensor Input Channel
    \item  0 - VCC ADC Input Voltage Range
\end{enumerate}

Suprisingly, there is a temperature sensor built into the MCU and it can be accessed via a single ended temperature sensor channel (page 351) \cite{ATMEGA328PB}. However, this sensor obviously measures chip temperature, not does it allow external sensor connection. Therefore this feature is of no use for the study.

As it is displayed in the list above, \textbf{Requirements 4b and 5b are satisified.}

