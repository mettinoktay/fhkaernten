\chapter*{Introduction}
\addcontentsline{toc}{chapter}{Introduction}
\par
3D printers have taken over the world over the last decade and
they have
found use in industry, as well as in homes. The convenience of producing
plastic parts at homes gathered attention of many hobbyists and makers. In
industry, the increasing production speed of 3D printers has made
them suitable for rapid-prototyping, saving immense amount of time and
money. Civil engineers and architects also found use
cases for 3D printers. For example, Mayorship of Istanbul has built its own
3D printed
local service buildings within a week with a fraction of cost and build
complexity.

This study will focus on FDM printers in which plastic filaments are
employed as raw material.

\subsection{Definitions}

\textbf{FDM Printer}: \textit{Fused Deposition Modeling} printers melt
plastic to fuse them onto each other. Combined with 3 or more axis
movement, this machine produces useful parts or ornaments out of plastic
filaments.

\textbf{PTC Thermistor:} A special kind of resistor whose resistance value
changes depending on its temperature. PTC stands for \textit{positive
temperature coefficient}, which indicates that the resistance of thermistor
is proportional to its temperature.

\textbf{Hotend:} It is the component that reaches up to 260 degree Celcius
to melt the plastic and deposit it. The temperature sensor is placed onto
this component.

\textbf{Nozzle:} It is a subcomponent of hotend, through which the molten
plastic meets its final destination.

\subsection{Need For Temperature Sensor}

Hotend temperature must be kept stable as fluctuations in hotend
temperature would cause quality issues on the product, e.g. uneven surface
finish and/or material deposition, or nozzle clog. Therefore, the frequency
at which the temperature is measured plays an important role. Also, since
the sensor is placed at the carriage, its weight plays an important role in
position control. However, a PTC thermistor is never of high mass,
therefore this study will assume that the mass of the carriage will remain
relatively same.

\subsection{Requirements - Boundary Conditions}

\begin{itemize}

     \item Measurement range: 160 - 270 \degree C
     \item Size: 2x2x2 mm
     \item Weight < 2 grams
     \item Accuracy $\pm$ 1 \degree C
     \item Power consumption < 2 mW
\end{itemize}

