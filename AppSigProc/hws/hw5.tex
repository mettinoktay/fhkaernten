\documentclass[12pt,a4paper]{report}
\usepackage[utf8]{inputenc}
\usepackage[T1]{fontenc}
\author{Metin Oktay Yılmaz}
\pagenumbering{gobble}
\begin{document}
	\textbf{Applied Signal Processing Homework \#5}

	\vspace{1cm}
	\textbf{Metin Oktay Yılmaz}\\

	\textit{Since I have no access to a scanner at the time of preparation and my phone's camera is bad, I had to prepare the homework on computer.} \\

	\textbf{1.} Determine the number range for a 16 bit number in:
	\begin{itemize}
		\item[a.]  2's complement fixed point format: \\

		Assuming $X_m = 1$:

		$$ -1 \le x \le (1-2^{16})$$
		$$ -1 \le x \le 0.999984741211$$

		\item[b.] Floating point format: \\
		$$-2^{31}(1-2^{-10}) \le x \le 2^{31}(1-2^{-10})$$
	\end{itemize}

	\textbf{2.}
	\begin{itemize}
		\item[a.] 5 bits can represent numbers between -16 and +15. The device that stores the coefficients should be upgraded to a 5 bits or more model.
		\item[b.] The LSB can not be a fixed bit, can it? It can change depending on the state of the system. To reduce the quantisation error, the 256 levels of quantisation should be assigned to a short range of numbers. Shorter the range, smaller the quantisation error.
		\item[c.] Rounding means to increase the real number to the next larger integer and remove the decimal ($0.9 \rightarrow 1.0$); truncation is directly removing the decimal part ($0.9 \rightarrow 0.0$) .
	\end{itemize}

\end{document}