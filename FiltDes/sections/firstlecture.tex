\section{Sampling Process and Z-Transformation}

\textbf{07.03.2023 - Lecture 1 - EDV3}

$$x(t) = \sum_{k=0}^{\infty}x[kT]* \delta(t-kT)$$

Sampling means that a signal from analog world is transferred into the digital world by means of periodic measurement and saving-to-memory.

Laplace Transform is as follows:

\begin{align*}
	X(s) &= \sum_{k=0}^{\infty}x[kT]*e^{-kTs} \\
	x(t) &= x[0]\delta(t) + x[1]\delta(t-T) + x[2]\delta(t-2T) ... \\
	X(s) &= X[0] + X[1]e^{-sT} + X[2]e^{-2sT} + ... \\
	   z &= e^{sT} \\
	X[z] &= X[0] + X[1]z^{-1} + X[2]z^{-2} \\
\end{align*}

\textbf{Note:} In z-plane, for stability, the roots must not rely outside of the unit circle (similar to s-plane and poles must rely on the left-hand side plane).

\section{Filter Implementations: FIR Filter}

A simple form of Finite Impulse Response filter is as follows:

$$H(z) = c_0 +c_1z^{-1}+c_2z^{-2} = \frac{z^{2}c_0 +zc_1+c_2}{z^2}$$


Since there is no feedback, \hl{an FIR filter is always stable}. All poles are located at the origin.

$$ \frac{Y(z)}{X(z)} = H(z) = c_0 + c_1z^{-1}+c_2z^{-2} $$

$$ Y(Z) = X(z)(c_0 + c_1z^{-1}+c_2z^{-2}) $$

Transform the system from z-domain into time domain as follows:

$$y[k] = c_0x[k] + c_1x[k-1] + c_2x[k-2]$$

\textbf{Example:}

$$ H(z) = c_0 + c_1z^{-1} + c_3z^{-3} $$

\begin{itemize}
	\item[1.] What is the difference equation?
\end{itemize}
$$y[k] = c_0x[k] + c_1x[k-1] + c_3x[k-3]$$

\begin{itemize}
	\item[2.] Represent the equation graphically.
\end{itemize}

It is the same process that we went through in today's Applied Signal Processing class.

\textbf{Example:}

$$y[k] = 3x[k] + 2x[k-1] - x[k-2]$$

\begin{itemize}
	\item[1.] Represent the equation graphically.
\end{itemize}

\begin{itemize}
	\item[2.] Output sequence for input $x[0] = 2$, $x[1] = 3$, $x[2] = 1$ and zero initial conditions.
\end{itemize}

\textbf{Example:}

$$y[k] = -x[k-2] + 2x[k-1] + 3x[k]$$

\begin{itemize}
	\item[1.] Output sequence for input $x[0] = 2$, $x[1] = 3$, $x[2] = 1$ and zero initial conditions.
\end{itemize}




