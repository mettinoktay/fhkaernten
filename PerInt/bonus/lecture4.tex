\documentclass[12pt,a4paper]{letter}
\usepackage[latin1]{inputenc}
\pagenumbering{gobble}

\begin{document}
	\textbf{30.03.2023 - Peripheral Interfacing}

	\textbf{Bonus Task 4}\\

	\textbf{Question:} Elaborate on measures against heat accumulation on PCBs.

	\textbf{Answer:} Heat generation is inevitable. The following are few of the ways to tackle the risk of excessive temperatures on PCBs:

	\begin{itemize}
		\item Passive-Active Cooling: This is the cheapest and easiest way to dissipate the generated heat. Placing an aluminum on top of high current components will help reduce the temperature. Adding a heat transfering medium between the heat sink and the component will increase dissipation rate, resulting in improved performance and life expectancy. When a heat sink is coupled with a blowing fan or a liquid cooler, its sinking capabilities increases further; increasing both performance, life expectancy and cost.

		\item Adjusting trace volume: Driving a motor, a light, or radio signal generator draw large currents, and consequently generate large amounts of heat. To protect the routes and the PCB, traces must be kept reasonably large (within prduction limits) to increase the heat dissipation. Otherwise, high temperatures could cause traces to break off of the prepreg, rendering the PCB useless.

		\item Thermal vias: By adding thermal vias, the PCB itself is used as a passive cooler. The effect is not close to using an actual heat-sink. However, this application may be advantageous where there are size limits.

	\end{itemize}

\end{document}