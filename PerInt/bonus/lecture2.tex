\documentclass[12pt,a4paper]{letter}
\usepackage[latin1]{inputenc}
\pagenumbering{gobble}

\begin{document}
	\textbf{09.03.2023 - Peripheral Interfacing}

	\textbf{Bonus Task 2}\\

	\textbf{Question:} Elaborate on the topic of embedded system dependability. How would an engineer know if an embedded system is dependable?

	\textbf{Answer:} To depend means \textit{to place reliance or trust.} For a system to be called dependable, it should rarely remind its existence. That means, the system must:

	\begin{itemize}
		\item handle exceptional occurences. For example, an unexpected physical or electrical shock during its operation should not cause a halt; e.g. the embedded system should be mounted on a shock-absorbing housing, or high voltage protection circuit on its power input terminals.
		\item handle natural elements. That is, its components should be corrosion proof or waterproof.
		\item handle component failures. That is, the components under heavy use, like a sensor or a driver, should be redundant.
		\item handle every situation that it was marketed to handle. This means that its firmware should be sophisticated, well- and thoroughly-tested, and if possible maintained after sale.
		\item alert its user or operator when an exception that is beyond its handling capabilities occurs. The number of such alerts must also be low.
	\end{itemize}

\end{document}