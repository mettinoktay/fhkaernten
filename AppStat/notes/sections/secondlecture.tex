\section{Something}

Something something - arrived 16 minutes late.

We take the time from the first exp. That would be minutes. Best set of factors with "\ofat" is:

$t = 130$ mins; $T = 225$ \degree C

If I start now and check the time at 200 \degree C.

Basic idea of DoE: paradigm change

Target oriented method instead of \ofat for onderstanding (or optimising) the process or product.

Slogan: Investigate many factors $x_1, x_2, x_3, ... , x_n$ (input variables) with few levels instead of few factors with many levels. Therefore we get the following DoE procedure:

\begin{itemize}
	\item[1] Recognise and describe the problem
	\item[2] Decide on the input parameters (factors) and their levels
	\item[3] Choose the appropriate response variable or variables
	\item[4] Design the experiment
	\item[5] Perform the experiment and measure the outcome of each run
	\item[6] Evaluate the dataset
	\item[7] Finalize and document the results
\end{itemize}

\section{Factorial Designs}

Let's say we have two factors $x_1, x_2$. We now choose the levels: Two levels for $x_1$, three levels for $x_2$. Altogether we have 6 combinations, which means 6 runs. This is called a $2*3$ factorial design. In general we have more than 2 factors: $x_1, x_2, x_3 ... x_p$

\begin{center}

	$x_1$: $l_1$ levels\\
	$x_2$: $l_2$ levels\\
	$x_3$: $l_3$ levels\\
	\dots \\
	$x_p$: $l_p$ levels\\

\end{center}

We have in total:
	$$l_1*l_2*...*l_p$$

factorial design. A speecial case is that each factor is varied on exactly two levels. This is called $2^p$ factorial design. \\

\textbf{Example:}\\

$p=2$ (2 factors) at 2 levels $\rightarrow 2^2$ factorial design\\

\begin{center}
	\begin{tabular}{|c|c|c|c|c|c|c|}
		\hline
		\rule[-1ex]{0pt}{2.5ex} num & factor $x_1$ & factor $x_2$ & mass flow  & temperature & depos. rate & notation \\
		\hline
		\rule[-1ex]{0pt}{2.5ex} 1 & - & - & 1000 sccm & 680 \degree C & 140 nm/min & $y_1$ \\
		\hline
		\rule[-1ex]{0pt}{2.5ex} 2 & + & - & 1500 sccm & 680 \degree C & 372 nm/min & $y_2$ \\
		\hline
		\rule[-1ex]{0pt}{2.5ex} 3 & - & + & 1000 sccm & 750 \degree C & 428 nm/min & $y_3$ \\
		\hline
		\rule[-1ex]{0pt}{2.5ex} 4 & + & + & 1500 sccm & 750 \degree C & 500 nm/min & $y_4$ \\
		\hline
	\end{tabular}
\end{center}

TODO: Make a massfow-temperature cartesian graph and point out the deposition rates.

%\begin{center}
%	\begin{tikzpicture}
%		\begin{axis}[
%			axis lines = left,
%			xlabel = \(x\),
%			ylabel = {\(f(x)\)},
%			]
%			%Below the red parabola is defined
%			\addplot [
%			domain=-10:10,
%			samples=100,
%			color=red,
%			]
%			{2};
%			\addlegendentry{\(x^2 - 2x - 1\)}
%			%Here the blue parabola is defined
%			\addplot [
%			domain=-10:10,
%			samples=100,
%			color=blue,
%			]
%			{1};
%			\addlegendentry{\(x^2 + 2x + 1\)}
%		\end{axis}
%	\end{tikzpicture}
%\end{center}

Main effect mass flow:

TODO correct this: $$\frac{500+372}{2} - \frac{2}{1}$$

Main effect temperature:

$$\frac{428 + 500}{2} - \frac{140+372}{2} = 208$$

If we change the temperature from 680 \degree C to 750 \degree C, the deposition rate changes by 208 nm/min, averaged over the mass flow. \\

3) What is the effect of the temperature on the deposition rate, if we change the temperature from \degree C to 750 \degree C and differentiate between the massflow.

Interaction between temperature and massflow (aka interaction temperature * massflow)

$$\frac{128-288}{2} = -80 = \frac{(y_4-y_2) - (y_3-y_1)}{2} = \frac{y_1+y_4}{2} - \frac{y_2+y_3}{2}$$

If we change the temperature from 680 \degree C to 750 \degree C and differentiate between the massflow, the deposition rate is changed by -80 nm/min \\

4) What is the effect of massflow on the deposition rate if we change the massflow from 1000 sccm to 1500 sccm and differentiate between the temperature?

$$\frac{72-232}{2} = -80 = \frac{(y_4-y_3) - (y_2-y_1)}{2} = \frac{y_1+y_4}{2} - \frac{y_2+y_3}{2}$$

If we change the massflow from 1000 sccm to 1500 sccm and differentiate between the temperature, the deposition rate is again changed by -80 nm/min. \\

If the temperature range of 680 \degree C and 750 \degree C and the mass flow range of 1000 sccm and 1500 sccm, the temperature has the strongest effect on the deposition rate, followed by the mass flow and the interaction.

We can now visualize the results:

Main effect plots:

TODO: Draw two cartesian graphs. one is massflow [sccm] - deposition rate [nm/min] the other is temperature [\degree C] - deposition rate [nm/min]
\begin{center}

	First plot:\\
	1000 sccm - 281 nm/min\\
	1500 sccm - 436 nm/min

	Main effect mass flow: $\Delta = 152$

	Second plot:\\
	680 \degree C - 256 nm/min\\
	750 \degree C - 464 nm/min

	Main effect temperature: $\Delta = 208$
\end{center}

there are many graphs drawn here. got lost.

cornerstone software is used to analyse the dataset.