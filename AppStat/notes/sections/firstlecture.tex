\section{Design of Experiment}

\textbf{07.03.2023 - Lecture 1 - EDV1}

\subsection*{What is a \textit{Design of Experiment}?}

It is a straightforward method which:
\begin{itemize}
	\item is efficient - that is, it has small number of tests
	\item is matched to the questions to be answered
	\item results in the highest possibşe precision
\end{itemize}

\textbf{Example:} How do you determine the weight of two objects?

\textbf{Method 1:} 2 measurements: First put object A on a scale and measure its weight. Then put object B on the scale and measure its weight.

\textbf{Method 2:} 2 measurements: Put object A and B at the same time on the scale and measure their combined weight. Then take one of the objects off of the scale and measure the weight of the object. Then subtract the two values to determine the weight of other object.

Method 2 is preffered for the following reasons:
\begin{itemize}
	\item Measurements never give the true values since they contain an uncertainty (\textit{s}). This phenomena is not unavoidable.
	\item The uncertainties (errors) occur randomly: Probability that the measurements are greater than the real value is equal to the probability that measurements are lower than the real value.
\end{itemize}

\textbf{Uncertainty of Method 1:} Chance that measured value is greater than A (that is, A + \textit{s}) is the same as it is lower than A (that is, A - \textit{s}).


\textit{\textbf{Expected uncertainty (Variance) $s$ of A:}}

$$0.5(A + s - A)^2 + 0.5 (A-s-A)^2 = s^2$$

\textit{\textbf{Result:}} Weight of A is known to $\pm s$

\textbf{Uncertainty of Method 2:} Now we have 4 possibilities for the outcome of the measurement:

\begin{itemize}
	\item $A+B$ and $A-B$ are both too big
	\item $A+B$ and $A-B$ are both too low
	\item $A+B$ is too low, $A-B$ is too big
	\item $A+B$ is too big $A-B$ is too low
\end{itemize}

\textit{\textbf{Expected uncertainty (Variance) of A:}}

$$ = 0.25* \bigg(\frac{(A+B+s)+(A-B+s)}{2} - A\bigg)^2 $$

$$ + 0.25* \bigg(\frac{(A+B-s)+(A-B-s)}{2} - A\bigg)^2 $$

$$ + 0.25* \bigg(\frac{(A+B-s)+(A-B+s)}{2} - A\bigg)^2 $$

$$ + 0.25* \bigg(\frac{(A+B+s)+(A-B-s)}{2} - A\bigg)^2 $$

$$ = 0.5s^2$$

So, the expected uncertainty of $A$ is now only $0.5s^2$. The weight of A is known to $\nicefrac{\pm s}{\sqrt{2}}$.

Method 1 is so-called one-factor-at-a-time measurement; in Method 2, every weight was effectively measured twice. Therefore we can say that in o DoE, in a statistical design experiment all factors are varied simultaneously; not one factor at a time.

\subsection*{Why would you \textit{design an experiment}?}

\begin{itemize}
	\item Method gives us a structured plan.
	\item Statistical experimental designs are aligned to statistical analysis(and the used software).
	\item Statistical experimental designs are much more efficient.
	\item Because of the structure plan, we are forced to get organized.
\end{itemize}

\subsection*{What is an \textit{efficient experiment}?}

\begin{itemize}
	\item Efficient experiment gets the required information at the least expenditure of resources.
\end{itemize}

\textbf{\textit{required}:} Not too much, not too little. Just right.

\textbf{\textit{least resources}:} Money, human resources, time-to-market.

\textbf{\textit{experiment and test}:} There is a big difference between test and experiment. A test is required to determine if some thing or things work or not. Experiment connects the "\textit{if}" of testing to determine "\textit{why}." Experiment is a structured set of coherent tests that are analysed as a whole to gain understanding of the process. Only with understanding, we are able to control the process. Knowing the functional relationship between input and output, we are able to develop robust products and processes with no suprises.

Prerequisites for a good experimentation are:

\begin{itemize}
	\item Knowledge of the process
	\item Having clear goals and objectives
	\item A response variable or variables (or namely outputs)
\end{itemize}

Knowledge of the process:
\begin{itemize}
	\item Prior knowledge from university or job training
	\item simulation of the process
	\item small preliminary experiments and/or tests
	\item other accumulated data
\end{itemize}

Having clear goals and objectives:
\begin{itemize}
	\item know the difference between goal and objectives. A goal is the destination that we will arrive after passing objectives.
\end{itemize}

A good response variable must be:
\begin{itemize}
	\item quantitative - that is, measurable
	\item precise - that is, aiming for a low range of deviation from target
	\item meaningul - that is, related to the customer's requirements
\end{itemize}

