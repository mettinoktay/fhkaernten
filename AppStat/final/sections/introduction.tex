\section{Introduction}
After many hours of theoretical study, it is time to put these attained knowledge into practical use.

The setup under scrutiny is provided as the following schema:

\begin{figure}[H]
	\centering
	\includegraphics[scale=0.5]{setup}
	\caption{\small Overview of the setup}
\end{figure}

\subsection{Characteristics of the Setup}

\begin{enumerate}
\item It consists of 2 tanks \tank{101} and \tank{102}, 2 valves \valve{104} and \valve{112}, and 1 pump.
\item The tanks are separated vertically, i.e. they have different potential energies. However, valves are placed to the same height.
\item \valve{112} valve connects two tanks together; while \valve{104} valve connects pump's output directly back to \tank{101} tank.
\item Valves are on either \ON or \OFF position. Pump motor operates at either 5V or 10V. Lastly, the level of \tank{102} is either 50-70mm or 240-260mm ...
\item ... therefore, there are 4 predictors in the setup: Each valve's position, pump motor voltage, and the water level of \tank{102}.
\end{enumerate}

The experiment was run twice to demonstrate reproducability. The results for each run is given in the next page.

\newgeometry{top=2cm,bottom=2cm,left=2cm,right=2cm}
\begin{table}[!h]
\begin{center}
    \def\arraystretch{1.2}
\begin{tabular}{||c|c|c|c|c|c|c|c||}
\hline
 & \valve{112} & \valve{104} & Pump Voltage & Level \tank{102} & Tendency & Time (s) & Flow Rate ($\nicefrac{mm^3}{s}$) \\
\hline
1 & \OFF & \OFF & 5V & 50-70 mm & rising & 19.6 & 218112 \\
\hline
2 & \ON & \OFF & 5V & 50-70 mm & no & $\infty$ & 0 \\
\hline
3 & \OFF & \ON & 5V & 50-70 mm & no & $\infty$ & 0 \\
\hline
4 & \ON & \ON & 5V & 50-70 mm & falling & 20.5 & -208536 \\
\hline
5 & \OFF & \OFF & 10V & 50-70 mm & rising & 9.9 & 431818 \\
\hline
6 & \ON & \OFF & 10V & 50-70 mm & rising & 21.4 & 199766 \\
\hline
7 & \OFF & \ON & 10V & 50-70 mm & rising & 17.2 & 234546 \\
\hline
8 & \ON & \ON & 10V & 50-70 mm & no & $\infty$ & 0 \\
\hline
9 & \OFF & \OFF & 5V & 240-260 mm & rising & 29.1 & 146907 \\
\hline
10 & \ON & \OFF & 5V & 240-260 mm & falling & 11.5 & -371739 \\
\hline
11 & \OFF & \ON & 5V & 240-260 mm & falling & 12.6 & -339285 \\
\hline
12 & \ON & \ON & 5V & 240-260 mm & falling & 5.2 & -822115 \\
\hline
13 & \OFF & \OFF & 10V & 240-260 mm & rising & 10.1 & 423267 \\
\hline
14 & \ON & \OFF & 10V & 240-260 mm & falling & 24.8 & -172379 \\
\hline
15 & \OFF & \ON & 10V & 240-260 mm & falling & 32.4 & -131944 \\
\hline
16 & \ON & \ON & 10V & 240-260 mm & falling & 5.5 & -777272 \\
\hline
\end{tabular}
\end{center}
\caption{First Run}
\end{table}
\begin{table}[!h]
\begin{center}
    \def\arraystretch{1.2}
\begin{tabular}{||c|c|c|c|c|c|c|c||}
\hline
 & \valve{112} & \valve{104} & Pump Voltage & Level \tank{102}  & Tendency & Time (s) & Flow Rate ($\nicefrac{mm^3}{s}$) \\
\hline
1 & \OFF & \OFF & 5V & 50-70 mm & rising & 19.1 & 223821\\
\hline
2 & \ON & \OFF & 5V & 50-70 mm & no & $\infty$ & 0 \\
\hline
3 & \OFF & \ON & 5V & 50-70 mm & no & $\infty$ & 0 \\
\hline
4 & \ON & \ON & 5V & 50-70 mm & falling & 30.0 & -142500 \\
\hline
5 & \OFF & \OFF & 10V & 50-70 mm & rising & 10.3 & 415048\\
\hline
6 & \ON & \OFF & 10V & 50-70 mm & rising & 19.6 & 218112 \\
\hline
7 & \OFF & \ON & 10V & 50-70 mm & rising & 22.7 & 188325 \\
\hline
8 & \ON & \ON & 10V & 50-70 mm & no & $\infty$ & 0 \\
\hline
9 & \OFF & \OFF & 5V & 240-260 mm & rising & 25.9 & 165057 \\
\hline
10 & \ON & \OFF & 5V & 240-260 mm & falling & 12.3 & -347560 \\
\hline
11 & \OFF & \ON & 5V & 240-260 mm & falling & 12.0 & -356250 \\
\hline
12 & \ON & \ON & 5V & 240-260 mm & falling & 4.8 & -890625 \\
\hline
13 & \OFF & \OFF & 10V & 240-260 mm & rising & 13.0 & 328846 \\
\hline
14 & \ON & \OFF & 10V & 240-260 mm & falling & 24.4 & -175204 \\
\hline
15 & \OFF & \ON & 10V & 240-260 mm & falling & 29.4 & -145408 \\
\hline
16 & \ON & \ON & 10V & 240-260 mm & falling & 5.8 & -737068 \\
\hline
\end{tabular}
\end{center}
\caption{Second run}
\end{table}
\restoregeometry

\subsection{Tasks}
\begin{enumerate}
\item Calculate the flow rate to or from \tank{102} for each run. The cross section of the tanks are 15 x 15 cm.
\item Evalute only the first run.
\item Fractional design of the first run (reduction by factor 2); evaluation of the reduced design and comparison with the full first run. Give detail about the lost information.
\item Common evaluation of first and second run.
\item Manually estimate the effects error by using the information of the replication. Elaborate on the significant effects. Interprete the results.
\item Additional evaluations, e.g. quality assessment of the model ... Use the evaluations from the lecture as a possible reference.
\end{enumerate}

\subsection{Calculation of Flow Rate}
The cross section is 15 x 15 cm and the flow rate is requested in $\nicefrac{mm^3}{s}$. Therefore the area is:
\begin{equation}
150\:mm \cdot 150\:mm = 22500\:mm^2
\end{equation}
To calculate the volume, depth is needed. The subtraction of levels will yield in depth. Higher thresholds are taken into account. Therefore the depth is:
\begin{equation}
260\:mm - 70\:mm = 190\:mm
\end{equation}
The total volume is then:
\begin{equation}
22500\:mm^2 \cdot 190\:mm = 4.275.000\:mm^3
\end{equation}
(3) was divided by each Time (s) value and appended to the tables above. The result is the flow rate for each run.

