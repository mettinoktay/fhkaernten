\newpage
\section{Evaluation of The First Run}

The position of V112 has the most positive effect to the flow rate, averaging around 400.000 $\nicefrac{mm^3}{sec}$, followed by the pump voltage, which averages around half the V112. The other two affected flow rate significantly negatively, and they both are rated around -350.000 $\nicefrac{mm^3}{sec}$.

It is intuitive to think that when the V112 is closed, the level of B102 is bound to increase. This opinion is correct when V104 is also OFF. (runs \#1, \#5, \#9, \#13). However, as the incoming pipe is located below the high water level mark, level of B102 becomes the deciding factor of the flow rate:

\begin{itemize}[noitemsep, topsep=0pt]
\item if it is between 50-70 mm, the flow rate is either 0 or positive (runs \#3, \#7)
\item else, it is always negative (i.e. tendency is falling). (runs \#11, \#15).
\end{itemize}

When the V112 is open, the tendecy depends on all the other factors. Explaining every case is time-consuming and uninteresting, therefore it will be omitted.

Following table and two graphs demonstrate the outcome of the first run:

\begin{table}[!h]
    \begin{center}
        \begin{tabular}{r|l}
            effect & estimate\\
            \cmidrule{1-2}
            average & -35014 \\
            \cmidrule{1-1}
            main effects &  \\
            \cmidrule{1-1}
            \valve{112} & 391962 \\
            \valve{104}& 198057 \\
            Pump voltage& -365033 \\
            Level \tank{102}& -365044 \\
        \end{tabular}
        \caption{Effects estimate}
    \end{center}
\end{table}

\begin{figure}[!h]
	\begin{center}
		\includegraphics[scale=0.4]{effect_pareto}
	\end{center}
	\caption{Effects Pareto}
\end{figure}

\begin{figure}[!h]
	\begin{center}
		\includegraphics[scale=0.3]{effect_probability_plot}
	\end{center}
	\caption{Effects Probability Plot}
\end{figure}
